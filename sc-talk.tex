%%for slideshow
\documentclass[ignorenonframetext]{beamer} %add option 'draft' for quicker compilation
\usepackage{beamerthemesplit,amsmath,amssymb,multirow}
\newcommand{\slides}{1}
%this only compiles frames with [label=current]
%\includeonlyframes{current} 
 
%for handouts
%\documentclass[a4paper,12pt]{article}
%\usepackage{beamerarticle}
%\newcommand{\slides}{0}

\mode<presentation>
{
%    \usetheme{AnnArbor} % horrible blue orange yellow....
%  \usetheme{Darmstadt} %blue one with balls and gradients
 % \usecolortheme{seahorse}  %% This looks horrible, pastel blue colours
\usecolortheme{rose}  %% Standard blue one that everyone uses
  \setbeamercovered{transparent}
  %Add university logo
  %\pgfdeclareimage[height=1cm]{university-logo}{logo}
  %\logo{\pgfuseimage{university-logo}}
  % If you wish to uncover everything in a step-wise fashion, uncomment
  % the following command:
%  \beamerdefaultoverlayspecification{<+->}
	\newcommand{\tcr}{\textcolor{red}}
}
\mode<article>{
  \usepackage{fullpage}
	\newcommand{\tcr}{\textcolor{black}}
  \renewcommand{\baselinestretch}{1.0}
  \oddsidemargin -1.5cm \evensidemargin -1.5in
  \topmargin=-1.5cm \headheight=0pt
  \headsep 0pt \textwidth=19cm
  \textheight=27cm \columnsep 10pt \columnseprule 0pt \parindent 0pt
  \parskip 0.0pt
  \usepackage{tweaklist}
  \renewcommand{\itemhook}{
    \setlength{\topsep}{-0pt}
    \setlength{\itemsep}{-0pt}
    \setlength{\parsep}{-0pt}
  }
  \renewcommand{\enumhook}{
    \setlength{\topsep}{-0pt}
    \setlength{\itemsep}{-0pt}
    \setlength{\parsep}{-0pt}
  }
}

\usepackage[english]{babel}
\usepackage[latin1]{inputenc}
\usepackage{times}
\usepackage[T1]{fontenc}
\usepackage{graphics,graphicx,fancyhdr,color,amsmath,url,enumerate,alltt}
\usepackage{epsf}
\usepackage{ifthen}


\newcommand{\bc}{\begin{center}}
\newcommand{\ec}{\end{center}}
\newcommand{\bn}{\begin{enumerate}}
\newcommand{\en}{\end{enumerate}}
\newcommand{\bi}{\begin{itemize}}
\newcommand{\ei}{\end{itemize}}
\newcommand{\be}{\begin{eqnarray}}
\newcommand{\ee}{\end{eqnarray}}
\newcommand{\bes}{\begin{eqnarray*}}
\newcommand{\ees}{\end{eqnarray*}}


\title[Smoothing over complex regions]{Using the Schwarz-Christoffel transform to smooth over complex regions}

%\subtitle{A mixture model approach}

\author[Miller]{David Lawrence Miller}

\institute{Mathematical Sciences\\University of Bath}

\date[24 March 2009] {Research Students' Conference 2009, Lancaster}
% - Either use conference name or its abbreviation.
% - Not really informative to the audience, more for people (including
%   yourself) who are reading the slides online

% Delete this, if you do not want the table of contents to pop up at
% the beginning of each subsection:
\mode<presentation> {
%    \AtBeginSubsection[]
    \AtBeginSection[]
    {
    \begin{frame}<beamer>
        \frametitle{Outline}
        %\tableofcontents[currentsection,currentsubsection]
        \tableofcontents[currentsection]
    \end{frame}
    }
}
\begin{document}

\begin{frame}
  \titlepage
\end{frame}

\mode<article>{
\maketitle
}

\mode<presentation> {
\begin{frame}
  \frametitle{Outline}
  \tableofcontents %[pausesections]
  % You might wish to add the option [pausesections]
\end{frame}
}

\section{Smoothing over complex regions}

\subsection{Intro}

\begin{frame}
	\frametitle{Smoothing in 2 dimensions}
       \bi
         \item Have some geographical region and wish to find out something about the biological population in it. 
         \item Want to find something out about the response given its coordinates and other covariates (eg. habitat, size, sex, etc.) 
         \item This problem is relatively easy if the domain is simple.
        \ei
        \bc
%  PIC HERE
%              \includegraphics[height=1.75in]{muw}

       \ec
\end{frame}

\begin{frame}
	\frametitle{Smoothing over complex domains}
       \bi
         \item Smoothing of complex domains makes this a lot more difficult.
         \item Problem of leakage.
         \item Models need to incorporate information about the intrinsic structure of the domain.
         \item Euclidean distance doesn't always make sense.
       \ei
\end{frame}

\subsection{Solutions}

\begin{frame}
	\frametitle{Possible solutions to leakage problems}
       \bi
         \item FELSPLINE (Ramsay, (2002).)
         \item Within-area distance (Wang and Ranalli. (2007).)
         \item Soap film smoothers (Wood \emph{et al.} (2008).) 
         \item Domain morphing (me!)
        \ei
        \bc
              \includegraphics[height=1.75in]{muw}
       \ec
\end{frame}

\begin{frame}
	\frametitle{Why morph the domain?}
       \bi
         \item Takes into account within-area distance.
         \item Gives a known domain that is easy to smooth over.
         \item  
        \ei
        \bc
              \includegraphics[height=1.75in]{muw}
       \ec
\end{frame}


\section{Domain morphing with the Schwarz-Christoffel transform}

\begin{frame}
	\frametitle{The Schwarz-Christoffel transform}
       \bi
         \item Take a polygon in some domain $W$ and morph it to a new domain $W^*$.
         \item Do this by starting at the new domain and working back to the polygon.
         \item Example: add extra edges to a rectangle and then alter angles.
         \item Can draw a polygonal bounding box around some arbitrary shape.
        \ei
        \bc
              \includegraphics[height=1.75in]{muw}
       \ec
\end{frame}


\begin{frame}
	\frametitle{Schwarz-Christoffel algorithm}
       \bi
         \item Start with a rectangle.
         \item Add edges.
         \item Iteratively deform by changing angles.
         \item Continue until the new shape is identical to the polygon.
         \item We then have a function for the mapping, $\varphi(x,y)$.
         \item $\varphi(x,y)$ is a conformal mapping.
        \ei 
        \bc
              \includegraphics[height=1.75in]{muw}
       \ec
\end{frame}


\section{Simulation experiment}

\begin{frame}
	\frametitle{Ramsay's horseshoes}
       \bi
         \item Ramsay proposed a horseshoe test function that precisely demonstrates the problem. 
         \item An alternate function was also proposed.
         \item Simulation consisted of taking a sample from the functions, adding noise and then trying to fit a transformed model along with a soap film for comparison.
         \item Use a bounding box around the horseshoe.
        \ei
        \bc
              \includegraphics[height=1.75in]{muw}
       \ec
\end{frame}

\begin{frame}
	\frametitle{Results (I)}
       \bi
         \item Morphing the horseshoe shape still gives a slightly odd domain. 
         \item We can see, however that we are still doing better than before.
         \item Looking at line plots, we can see the difference in gradient.
      %% Line plots

        \ei
        \bc
              \includegraphics[height=1.75in]{muw}
       \ec
\end{frame}

\begin{frame}
	\frametitle{Results (II)}
       \bi
         \item 
      % table of MSE results?
         \item
         \item 
        \ei
        \bc
              \includegraphics[height=1.75in]{muw}
       \ec
\end{frame}
\begin{frame}
	\frametitle{Where next?}
       \bi
         \item 
         \item
         \item 
        \ei
        \bc
              \includegraphics[height=1.75in]{muw}
       \ec
\end{frame}


\begin{frame}
	\frametitle{References}
       \bi
         \item soap
         \item wang and ranalli
         \item trefethen
         \item ramsay
        \ei
        \bc
              \includegraphics[height=1.75in]{muw}
       \ec
\end{frame}


\end{document}
